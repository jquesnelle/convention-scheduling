\documentclass[]{article}
\usepackage[total={6in,9in},top=1in, left=1in, right=1in, bottom=1in]{geometry}
\usepackage{enumerate}
\usepackage{amssymb,amsmath,amsthm}
\usepackage{soul}
% \usepackage{mathtools} 

% ----------------------------------------------------------------
\vfuzz2pt % Don't report over-full v-boxes if over-edge is small
\hfuzz2pt % Don't report over-full h-boxes if over-edge is small
% THEOREMS -------------------------------------------------------
\newtheorem{thm}{Theorem}[section]
\newtheorem{cor}[thm]{Corollary}
\newtheorem{lem}[thm]{Lemma}
\newtheorem{prop}[thm]{Proposition}
\theoremstyle{definition}
\newtheorem{defn}[thm]{Definition}
\theoremstyle{remark}
\newtheorem{rem}[thm]{Remark}
\numberwithin{equation}{section}

%opening
\title{Undergrad research overview}
\author{Jeffrey Quesnelle}

\begin{document}

\maketitle

\section{Problem overview}
The primary problems we will be investigating are so-called scheduling algorithms. In general scheduling algorithms involve taking a description of some resources and assigning them to time slots given constraints. The particular problem we are interested in is motivated by the problem of scheduling presentations at a weekend convention. At a weekend convention there are a fixed amount of time slots available for presentations to take place in. All presentations must be scheduled. Each presentation is given by one or more presenters. During a presentation all presenters can only be at that specific presentation. Each presentation should be assigned one time slot and one room.

We are particularly interested in a system that allows attendees of the convention to indicate interest in attending specific presentations. When the scheduling algorithm is executed the algorithm should attempt to minimize the conflicts i.e. we want to maximize the number of presentations that attendees can go to (minimizing schedules that put presentations indicated by an attendee as being interested in from being scheduled in the same time slot in different rooms).

The motivator for this research is an open source convention scheduling web app being written by the author known as ``TuxTrax": https://github.com/MattArnold/penguicontrax. The app is being developed for Penguicon, an annual open source/science fiction/fantasy convention that takes place annually in Metro Detroit: http://penguicon.org/. 

\subsection{Problem definitions}

There are a few different versions of the problem we are interested in. This is not an exhaustive list and may be expanded as more research is performed.

\begin{quote}
	\textsc{Convention Timetable Decision Problem}
	
	\underline{INSTANCE}:
	\begin{enumerate}
		\item a finite set $H$ of hours;
		\item a collection $\{P_1, P_2, \cdots, P_n\}$, where $P_i \subseteq H$ (there are $n$ presenters and $P_i$ is the set of hours during which the $i$th teacher is available for teaching);
		\item a collection $\{T_1, T_2, \cdots, T_m\}$, where $T_j \subseteq H$ (there are $m$ talks and $T_j$ is the set of hours during which the $j$th talk can be given);
		\item an $n \times m$ matrix $G$ of nonnegative integers ($G_{ij}$ is the number of hours (times) which the $i$th presenter will give the $j$th talk).
	\end{enumerate}
	\underline{QUESTION}: Does there exist a meeting function 
	\begin{gather*}
		f(i,j,h) : \{1,\cdots,n\} \times \{1,\cdots,m\} \times H \rightarrow \{0,1\}
	\end{gather*}
	(where $f(i,j,h)=1$ if and only if presenter $i$ gives talk $j$ during hour $h$) such that
	\begin{enumerate}[(a)]
		\item $f(i,j,h) = 1 \Rightarrow h \in P_i \cap T_j$ (the presenter and talk are both available to be scheduled at hour $h$);
		\item $\sum\limits_{h \in H} f(i,j,h) = G_{ij}$ for all $1 \le i \le n$ and $1 \le j \le m$ (the $i$th presenter was scheduled for the $j$th talk the required number of times);
		\item $G_{ij} > 0 \land G_{i'j} > 0 \Rightarrow f(i,j,h)=f(i',j,h)$ for all $1 \le i,i' \le n$, $1 \le j \le m$ and $h \in H$ (all presenters that are required to give a talk must be present at all instances of that talk);
		\item $\sum\limits_{j=1}^m f(i,j,h) \le 1$ for all $1 \le i \le n$ and $h \in H$ (no presenter is giving more than one talk simultaneously).
	\end{enumerate}
\end{quote}
\begin{quote}
	\textsc{Extended Convention Timetable Decision Problem}
	
	\underline{INSTANCE}: Same as CTTD, but with the additional parameters:
		\begin{enumerate}[1.]
			\setcounter{enumi}{4}
			\item a finite set $R$ of rooms;
			\item a collection $\{A_1,A_2,\cdots,A_r\}$, where $A_k \subseteq H$ (there are $r=|R|$ rooms and $A_k$ is the set of hours during which the $k$th room is available);
			\item a collection $\{S_1,S_2,\cdots,S_m\}$, where $S_l \subseteq R$ (there are $m$ talks and $S_m$ is the set of rooms that the $l$th presentation may be given in)
		\end{enumerate}
	\underline{QUESTION}: Does there exist a meeting function 
		\begin{gather*}
			f(i,j,h,r) : \{1,\cdots,n\} \times \{1,\cdots,m\} \times H \times R \rightarrow \{0,1\}
		\end{gather*}
		(where $f(i,j,h,r)=1$ if and only if presenter $i$ gives talk $j$ during hour $h$ in room $r$) such that
		\begin{enumerate}[(a)]
			\item $f(i,j,h,r) = 1 \Rightarrow h \in P_i \cap T_j \cap A_r \land r \in S_j$ (the $i$th presenter, $j$th presentation and room $r$ are all available to be scheduled at hour $h$ and room $r$ is suitable for the $j$th presentation);
			\item $\sum\limits_{r \in R}\sum\limits_{h \in H} f(i,j,h,r) = G_{ij}$ for all $1 \le i \le n$ and $1 \le j \le m$ (the $i$th presenter was scheduled for the $j$th presentation the required number of times);
			\item $G_{ij} > 0 \land G_{i'j} > 0 \Rightarrow f(i,j,h,r)=f(i',j,h,r)$ for all $1 \le i,i' \le n$, $1 \le j \le m$, $h \in H$, and $r \in R$ (all presenters that are required to give a talk must be present at all instances of that talk);
			\item $\sum\limits_{r \in R}\sum\limits_{j=1}^m f(i,j,h,r) \le 1$ for all $1 \le i \le n$ and $h \in H$ (no presenter is giving more than one talk simultaneously).
		\end{enumerate}
\end{quote}

\section{Scope of research}
The major questions we hope to answer in the research are the following:
\begin{enumerate}[(1)]
	\item Is CTTD NP-Complete? Is ECTTD NP-Complete? If so are there certain relaxations to CTTD and ECTTD that will place them in P?
	\item What is an algorithm that both finds a solution to an instance of CTTD/ECTTD and minimizes conflicts from the preferences of attendees (known ahead of time)? Can we show that this algorithm is optimal (i.e. that there is no solution for the same problem set that has fewer conflicts)?
\end{enumerate}

We will use integer programming as the basis for our solution to Problem 2. Once a specific integer programming solution is found a possible stretch problem will be to find more and more efficient integer programming models.

\subsection{Current progress}
Some progress has already been made, mostly on Problem 2. Available to the author are the databases for two previous years of Penguicon: this means the names and presenters for each presentation. This data was used to ``seed" a database mimicking a future convention. Random users were added who randomly ``RSVPed" to presentations. C++ code was written to generate a .lp file which can be solved by 3rd-party linear solvers like CoinMP. The model that the code generates has not yet been proven to be correct or optimal. The code for the modeler can be found at https://github.com/MattArnold/penguicontrax/tree/master/modeler.

\section{Existing research}
Scheduling problems have been studied for at least forty years with a seminal treatment in \ul{Computers and Intractability: A Guide to the Theory of NP-Completeness} which gave the classical definition of the ``Timetable Decision Problem" (TTD). TTD was shown to be NP-Complete in \textit{On the complexity of timetable and multicommodity flow problems} (Even, S., A. Itai, and A. Shamir. 1976. SIAM Journal on Computing 5, (4) (12): 691-13). The recent paper \textit{On the complexity of scheduling university classes} (Lovelace, A. 2010. M.S. in Computer Science Thesis. California Polytechnic State University: U.S.A.) investigated similar problems as ours in the context of scheduling university classes. Interestingly enough Lovelace showed that certain versions of the scheduling problem are indeed in P.

\end{document}
