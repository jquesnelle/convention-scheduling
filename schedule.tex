\documentclass[]{article}
\usepackage[total={6in,9in},top=1in, left=1in, right=1in, bottom=1in]{geometry}
\usepackage{enumerate}
\usepackage{amssymb,amsmath,amsthm}
\usepackage{graphicx}

% ----------------------------------------------------------------
\vfuzz2pt % Don't report over-full v-boxes if over-edge is small
\hfuzz2pt % Don't report over-full h-boxes if over-edge is small
% THEOREMS -------------------------------------------------------
\newtheorem{thm}{Theorem}[section]
\newtheorem{cor}[thm]{Corollary}
\newtheorem{lem}[thm]{Lemma}
\newtheorem{prop}[thm]{Proposition}
\theoremstyle{definition}
\newtheorem{defn}[thm]{Definition}
\theoremstyle{remark}
\newtheorem{rem}[thm]{Remark}
\numberwithin{equation}{section}

%opening
\title{On the complexity of scheduling conventions}
\author{Jeffrey Quesnelle, Daniel Steffy}

\begin{document}

\maketitle

\begin{abstract}
The problem of determining if there is a feasible schedule for a convention given the constraints of rooms, presenters, and attendees is closely related to the so-called timetable problem, some formulations of which are NP-Complete with corresponding optimization problems that are NP-Hard. The complexity classes of various formulations of a convention scheduling problem are given as well as an investigation into models that optimally solve certain types of these problems with integer programming.
\end{abstract}

\section{Introduction}

The authors of this paper were working on developing a website for a local convention. Part of the scope for the website involved developing a scheduling portion; various talks and presenters needed to be scheduled into the limited programming space and time so as to ensure no presenter had dual-bookings. Furthermore, the website gave the ability for attendees to ``RSVP" to events before-hand, giving the extra complication of trying to generate the schedule that minimized conflicts amongst presentations attendees had previously indicated they were interested in.

\subsection{Background and related problems}
The problem of assigning resources to some tasks given a set of constraints is a problem that has been studied widely in literature and appears often in day-to-day life. A new university student perusing the catalog of scheduled classes would be right to marvel at the ingenuity it takes to align different professor skills, availabilities, and course requirements into a cohesive plan. (Although often their opinion on its ingenuity may vary when inevitably two required courses are only offered concurrently!)  

One of the oldest scheduling-like problem that has been studied is the \emph{timetable design problem} (TTD). TTD, informally, is the problem of determining if given the specific hours of the week, a set of teachers and their available teaching hours, and a matrix describing for each teacher which courses he is required to teach, there exists a schedule that satisfies the constraints. TTD was shown to be NP-Complete in 1976 via a 3-SAT reduction \cite{even76}. However, if certain special conditions are met within the restrictions then it is possible that TTD $\in$ P. For example, if each teacher is only available for up to two hours, or each teacher is able to teach any class, then the problem is solvable in polynomial time \cite{garey76}.

Unfortunately, classical TTD often doesn't map well onto several common problems such as scheduling courses for a university. Specifically, the requirement that a teacher \emph{must} teach certain classes may be relaxed to describing those classes they are \emph{willing} to teach. This is the Basic Course Scheduling problem (BCS). Perhaps surprisingly, BCS $\in$ P was shown by Lovelace, A. in 2010 \cite{lovelace2010}. Specifically, she first constructed an algorithm to find the optimal schedule, and then used this to give an affirmative yes/no answer to BCS by validating that all classes were indeed scheduled. It appears that our problem sits on the ``boundary" of P and NP; a slight modification of the requirements appears enough to tip it either way. For example, if we extend BCS to include the scheduling of rooms, then this problem (known as the Extended Course Scheduling problem or ECS) is not in P, and the corresponding problem of finding an optimal schedule for ECS is NP-Hard.

\section{Problem descriptions and classes}
\subsection{Timetable Decision Problem}
Although the previous discussion regarding BCS and ECS is illuminating, our specific problem of convention scheduling more closely resembles TTD. The precise formulation of TTD is as follows.
\begin{quote}
	\textsc{Timetable Decision Problem}
	
	\underline{INSTANCE}:
	\begin{enumerate}
		\item a finite set $H$ of hours;
		\item a collection $P = \{P_1, P_2, \cdots, P_n\}$, where $P_i \subseteq H$ (there are $n$ presenters and $P_i$ is the set of hours during which the $i$th presenter is available for presenting);
		\item a collection $T = \{T_1, T_2, \cdots, T_m\}$, where $T_j \subseteq H$ (there are $m$ talks and $T_j$ is the set of hours during which the $j$th talk can be given);
		\item an $n \times m$ matrix $G$ of nonnegative integers ($G_{ij}$ is the number of hours (times) which the $i$th presenter will give the $j$th talk).
	\end{enumerate}
	\underline{QUESTION}: Does there exist a function 
	\begin{gather*}
		f(i,j,h) : \{1,\cdots,n\} \times \{1,\cdots,m\} \times H \rightarrow \{0,1\}
	\end{gather*}
	(where $f(i,j,h)=1$ if and only if presenter $i$ gives talk $j$ during hour $h$) such that
	\begin{enumerate}[(a)]
		\item $f(i,j,h) = 1 \Rightarrow h \in P_i \cap T_j$ (the presenter and talk are both available to be scheduled at hour $h$);
		\item $\sum\limits_{h \in H} f(i,j,h) = G_{ij}$ for all $1 \le i \le n$ and $1 \le j \le m$ (the $i$th presenter was scheduled for the $j$th talk the required number of times);
		\item $\sum\limits_{i=1}^n f(i,j,h) \le 1$ for all $1 \le j \le m$ and $h \in H$ (no talk has more than one presenter at a time);
		\item $\sum\limits_{j=1}^m f(i,j,h) \le 1$ for all $1 \le i \le n$ and $h \in H$ (no presenter is giving more than one talk simultaneously).
	\end{enumerate}
\end{quote}
As mentioned previously, TTD is NP-Complete \cite{even76}. Now, consider the following restricted version TTD; if a presenter $i$ is called a $k$\textit{-presenter} when $|P_i|=k$ and is called \textit{tight} when $|P_i|=\sum\limits_{j=1}^{m} G_{ij}$ (i.e., presenting whenever he is available), then define the Restricted Timetable Decision problem (RTTD) as:
\begin{quote}
	\textsc{Restricted Timetable Decision Problem}
	
	Same as TTD, with the following additional restrictions:
	\begin{enumerate}
		\item $|H|=3$;
		\item $T_j = H$ for all $1 \le j \le m$; (talks are always available)
		\item each presenter is either a tight 2-presenter or a tight 3-presenter;
		\item $G_{ij}=0$ or $1$ for every $1 \le i \le n$ and $1 \le j \le m$.
	\end{enumerate}
\end{quote}
RTTD has been shown to be NP-Complete via a direct 3-SAT reduction \cite{even76} (in fact, TTD was shown to be NP-Complete by showing that RTTD is NP-Complete).

\subsection{Convention TTD Problem}
Although TTD solves an important portion of the problem, the requirements of a convention may be slightly different: a more common instance is that there may be multiple presenters for each talk. The restraint (c) of TTD explicitly forbid this; in particular, (c) ensures that each talk is scheduled to exactly one presenter. In the case where multiple presenters are allowed we most likely wish to add a different constraint: that for each talk \emph{every} presenter that can be scheduled is scheduled. As an illustration, if Alice is giving talks A, B, and C, and Bob is giving talks B, C, and D, all scheduled instances of B and C should include both Alice and Bob. This leaves us with the Convention Timetable Decision problem (CTTD).
\begin{quote}
	\textsc{Convention Timetable Decision Problem}
	
	Same as TTD, but with restraint (c) changed to
	\begin{enumerate}[(c)]
		\item $G_{ij} > 0 \land G_{i'j} > 0 \Rightarrow f(i,j,h)=f(i',j,h)$ for all $1 \le i,i' \le n$, $1 \le j \le m$ and $h \in H$ (all presenters that are required to give a talk must be present at all instances of that talk);
	\end{enumerate}
\end{quote}
Now that we have defined our problem it behooves us to determine what complexity class it is in. The problem of determining the chromatic number of a graph (i.e. the minimum number of different colors that can be assigned to a graph's vertices so that no two adjacent vertices share the same color) is NP-Hard and the corresponding decision problem asking if a certain graph admits a $k$-coloring is NP-Complete \cite{garey76_2}. We will provide a reduction of Graph $k$-Colorability to CTTD.
\begin{thm}
CTTD is NP-Complete.
\end{thm}
\begin{proof}\label{cttd_np}
We first note that CTTD $\in$ NP. The function $f$ is a certificate of size $nm|H|$, and the requirements on $f$ (a-d) are easily checked in polynomial time.

Let $G=(V,E),k$ be an instance of Graph $k$-Colorability. Let $H=\{1, 2, \cdots, k\}$. For each vertex $v_k \in V=\{v_1, v_2, \cdots, v_{|V|}\}$ we let $T_k = H$. For each edge $e_l \in E=\{e_1, e_2, \cdots, e_{|E|}\}$ we let $P_l = H$. Now, $e_l = (c,d)$ where $c,d \in V$. Let $c'$ be in the index of $c \in E$ and $d'$ be the index of $d \in E$. Let $G_{lc'} = 1$, $G_{ld'} = 1$, and let all other entries be zero. We now have an instance $(H, P, T, G)$ of CTTD (whose construction was easily computed in polynomial time). 

If and only if algorithm accepts this instance then there exists a function $f$ that meets the requirements (a-d). Without loss of generality we assume that the graph is connected (in the disconnected case we can solve the connected subgraph which will admit the same $k$-Colorability). Then, for any talk $T_j$ (corresponding to vertex $v_j$)  by requirement (b) there exists at least one $i, h$ such that $f(i,j,h) = 1$. If $v_j$ has more than one edge then there will exist another $i',h'$ such that $f(i',j,h')=1$ (recall that all entires of $G$ are either zero or one). But, requirement (c) ensures that in this case $h=h'$ and so for any $T_j$ there is only one $h$ and thus $v_j$ has color $h \in \{1,2,\cdots,k\}$ (since $T_j=P_i=H$ by requirement (a)). Finally, for any two scheduled talks $T_j$ and $T_{j'}$ with a common presenter $P_i$ (corresponding to adjacent vertices $v_j, v_{j'}$) each will have a different hour (color) $h$ by requirement (d).

In conclusion, any algorithm that accepts or rejects our instance of CTTD would have the same behvior as an algorithm for Graph $k$-Colorability, and since Graph $k$-Colorability is NP-Complete, CTTD is NP-Complete.
\end{proof}

\subsection{Basic Convention TTD Problem}
Lovelace showed that a relaxed version of TTD (called BCS for ``Basic Course Scheduling") can be solved in polynomial time using a network flow model \cite{lovelace2010}. The principal differences between BCS and TTD are the reduction of many ``hard" requirements such as those insisting that a presenter give \emph{exactly} a certain number of talks of a certain type to simply saying they may give \emph{at most} a number of talks for which they are \emph{willing} to give. BCS does maintain a hard requirement that all presentations must be scheduled (otherwise simply not holding the convention at all would be a feasible solution!), it simply is less picky on who exactly does the presenting. We formulate BCS in terms our previously used nomenclature as the Basic Timetable Decision Problem (BTTD).
\begin{quote}
	\textsc{Basic Timetable Decision Problem}
	
	\underline{INSTANCE}:
		\begin{enumerate}
			\item a finite set $H$ of hours;
			\item a collection $P = \{P_1, P_2, \cdots, P_n\}$, where $P_i \subseteq H$ (there are $n$ presenters and $P_i$ is the set of hours during which the $i$th presenter is available for presenting);
			\item a collection $T = \{T_1, T_2, \cdots, T_m\}$, where $T_j \subseteq H$ (there are $m$ talks and $T_j$ is the set of hours during which the $j$th talk can be given);
			\item a function $L : \mathbb Z^+ \rightarrow \mathbb Z_0^+$, where $L(n)$ is the maximum number of talks that the $n$th presenter can give;
			\item a function $S : \mathbb Z^+ \rightarrow \mathbb Z_0^+$, where $S(m)$ is the desired number of instances of the $m$th presentation;
			\item a function $WTP : \{1,2,\cdots, n\} \times \{1,2,\cdots,m\} \rightarrow \{0,1\}$, where $WTP(i,j)$ indicates if the $i$th presenter is Willing To Present the $j$th talk.
		\end{enumerate}
		\underline{QUESTION}: Does there exist a function 
		\begin{gather*}
			f(i,j,h) : \{1,\cdots,n\} \times \{1,\cdots,m\} \times H \rightarrow \{0,1\}
		\end{gather*}
		(where $f(i,j,h)=1$ if and only if presenter $i$ gives talk $j$ during hour $h$) such that
		\begin{enumerate}[(a)]
			\item $f(i,j,h) = 1 \Rightarrow h \in P_i \cap T_j$ (the presenter and talk are both available to be scheduled at hour $h$);
			\item $\sum\limits_{h \in H} f'(j,h) = S(j)$ for all $1 \le j \le m$ where $f'(j,h) = 1 \iff \exists i \text{ with } 1 \le i \le n$ such that $f(i,j,h)=1$, and 0 otherwise (the $j$th talk is given the required number of times);
			\item $\sum\limits_{j=1}^{m} f(i,j,h) \le 1$ for all $1 \le i \le n$ and $h \in H$ (there is no more than one presenter scheduled for each instance of a talk);
			\item $f(i,j,h) = 1 \Rightarrow WTP(i,j) = 1$ (only presenters willing to give a talk are scheduled for it);
			\item $\sum\limits_{j=1}^n\sum\limits_{h \in H} f(i,j,h) \le L(i)$ for all $1 \le i \le n$ (the total number of talks that the $i$th presenter is scheduled for is at most their maximum number of presentations)
			\item $\sum\limits_{j=1}^m f(i,j,h) \le 1$ for all $1 \le i \le n$ and $h \in H$ (no presenter is giving more than one talk simultaneously).
		\end{enumerate}
\end{quote}
\begin{rem}
BTTD $\in$ P \cite{lovelace2010}.
\end{rem}
The difference between TTD and CTTD is the ability for presentations to have multiple presenters, and the requirement that all presenters be scheduled for all instances of a talk. Likewise, we can formulate a modified version of BTTD that incorporates this new constraint which we will call the Basic Convention Timetable Decision problem (BCTTD). 
\begin{quote}
	\textsc{Basic Convention Timetable Decision Problem}
	
	Same as BTTD, but with constraint (c) changed to
	\begin{enumerate}[(c)]
		\item $WTP(i,j) = WTP(i',j) \Rightarrow f(i,j,h)=f(i',j,h)$ for all $1 \le i,i' \le n$, $1 \le j \le m$ and $h \in H$ (all presenters that are required to give a talk must be present at all instances of that talk);
	\end{enumerate}
\end{quote}
\begin{thm}
BCTTD is NP-Complete.
\end{thm}
\begin{proof}\label{bcttd_np}
Let $G=(V,E),k$ be an instance of Graph $k$-Colorability. Let $H=\{1, 2, \cdots, k\}$. For each vertex $v_k \in V=\{v_1, v_2, \cdots, v_{|V|}\}$ we let $T_k = H$. For each edge $e_l \in E=\{e_1, e_2, \cdots, e_{|E|}\}$ we let $P_l = H$. Now, $e_l = (c,d)$ where $c,d \in V$. Let $c'$ be in the index of $c \in E$ and $d'$ be the index of $d \in E$. Let $WTP(l,c')=1$ and $WTP(l,d')=1$. For all $1 \le i \le n$, let $L(i)=2$ and for all $1 \le j \le m$, let $S(j)=1$. We now have an instance $(H,P,T,L,S,WTP)$ of BCTTD.

If and only if algorithm accepts this instance then there exists a function $f$ that meets the requirements (a-f). As with CTTD we assume (without loss of generality) that $G$ is connected. Then, for any talk $T_j$ (corresponding to vertex $v_j$) by requirement (b) there exists at least one $i, h$ such that $f(i,j,h) = 1$ since $S(j)=1$ for all $1 \le j \le m$. If there exist any other $i',h'$ with $f(i',j,h')=1$ then this means that $P_{i'}$ is also scheduled for $T_j$ at $h'$, but since by requirement (d) this tells us that $WTP(i',j)=1$ (and we also have $WTP(i,j)=1$ for the same reason) and thus $WTP(i,j)=WTP(i',j)$ and requirement (c) gives us that $h=h'$ and so for any $T_j$ there is only one $h$ and thus $v_j$ has color $h \in \{1,2,\cdots,k\}$ (since $T_j=P_i=H$ by requirement (a)). For any two scheduled talks $T_j$ and $T_{j'}$ with a common presenter $P_i$ (corresponding to adjacent vertices $v_j, v_{j'}$) each will have a different hour (color) $h$ by requirement (f). Lastly we know that this formulation enumerates all presenters (edges) and talks (vertices) by requirement (e); in particular because $L(i) = 2$ and $\sum_{j=1}^{m} WTP(i,j) = 2$.
\end{proof}
Perhaps somewhat surprisingly the mere modification of requirement (c) in BTTD to our stronger version in BCTTD is enough to move the problem from P into NP-Complete. \\
\begin{center}
Figure 1: An illustration of a reduction from $k$-Colorability to BCTTD
\end{center}
\setcounter{figure}{1}
\underline{QUESTION}: Does the following graph admit a 3-color?
\begin{center}
\includegraphics[scale=0.75]{3colorinstance}
\end{center}
From the construction in the proof for Theorem \ref{bcttd_np}: \\
\begin{center}
\begin{tabular}{ l c r }
$H=\{1,2,3\}$ & $P=\{H,H,H,H,H\}$ & $T=\{H,H,H,H,H\}$
\end{tabular}
\begin{tabular}{ c c }
$WTP=$ \begin{tabular}{ r | c | c | c | c | c }
$v_1$ & 1 & 1 & 0 & 0 & 0 \\ \hline
$v_2$ & 1 & 0 & 1 & 0 & 0 \\ \hline
$v_3$ & 0 & 1 & 1 & 1 & 0 \\ \hline
$v_4$ & 0 & 0 & 0 & 1 & 1 \\ \hline
$v_5$ & 0 & 0 & 0 & 0 & 1 \\ \hline
      & $e_1$ & $e_2$ & $e_3$ & $e_4$ & $e_5$
\end{tabular} & 
\begin{tabular}{ c c }
$L=$ \begin{tabular}{ c | c | c | c | c }
2 & 2 & 2 & 2 & 2 \\ \hline
$e_1$ & $e_2$ & $e_3$ & $e_4$ & $e_5$
\end{tabular} & $S=$ \begin{tabular}{ c | c | c | c | c }
1 & 1 & 1 & 1 & 1 \\ \hline
$v_1$ & $v_2$ & $v_3$ & $v_4$ & $v_5$
\end{tabular}
\end{tabular} 
\end{tabular}
\end{center}
An accepting solution to BCTTD: \\
\begin{center}
$f=$
\begin{tabular}{ c c c }
\begin{tabular}{r | c | c | c | c | c}
$v_1$ & 1 & 1 & 0 & 0 & 0 \\ \hline
$v_2$ & 0 & 0 & 0 & 0 & 0 \\ \hline
$v_3$ & 0 & 0 & 0 & 0 & 0 \\ \hline
$v_4$ & 0 & 0 & 0 & 1 & 1 \\ \hline
$v_5$ & 0 & 0 & 0 & 0 & 0 \\ \hline
      & $e_1$ & $e_2$ & $e_3$ & $e_4$ & $e_5$
\end{tabular} & 
\begin{tabular}{r | c | c | c | c | c}
$v_1$ & 0 & 0 & 0 & 0 & 0 \\ \hline
$v_2$ & 1 & 0 & 1 & 0 & 0 \\ \hline
$v_3$ & 0 & 0 & 0 & 0 & 0 \\ \hline
$v_4$ & 0 & 0 & 0 & 0 & 0 \\ \hline
$v_5$ & 0 & 0 & 0 & 0 & 1 \\ \hline
      & $e_1$ & $e_2$ & $e_3$ & $e_4$ & $e_5$
\end{tabular} & 
\begin{tabular}{r | c | c | c | c | c}
$v_1$ & 0 & 0 & 0 & 0 & 0 \\ \hline
$v_2$ & 0 & 0 & 0 & 0 & 0 \\ \hline
$v_3$ & 0 & 1 & 1 & 1 & 0 \\ \hline
$v_4$ & 0 & 0 & 0 & 0 & 0 \\ \hline
$v_5$ & 0 & 0 & 0 & 0 & 0 \\ \hline
      & $e_1$ & $e_2$ & $e_3$ & $e_4$ & $e_5$
\end{tabular} \\
$h=1$ & $h=2$ & $h=3$
\end{tabular}
\end{center}
Which gives rise to the coloring (colors represented as filled, solid, and dashed vertices):
\begin{center}
\includegraphics[scale=0.75]{3colorinstancecolored}
\end{center}

\subsection{Extended Convention TTD Problem}
CTTD provides the ``core" of the scheduling problem we investigated; while it remains NP-Complete the problem is a significant enough deviation from canonical TTD that it required a formal proof of its NP-Completeness. We now present a modification to CTTD that consists mainly of additional constraints to certain parameters, but CTTD will be seen as a trivial special case.

CTTD ensures that each presenter is correctly scheduled for their presentations without conflict, but this may not be sufficient in a real-world scenario. A more useful solution would be one that not only schedules presenters and presentations but also assigns to each presentation a room in which the presentation will take place (ensuring that each presentation has exclusive access to its room at each hour). Furthermore, symmetric with the suitability sets in TTD that restricts the availability of presentations and presenters to certain times, rooms also may only be available during certain times or suitable for certain presentations, so this must be factored into the problem. To this end we present the Extended Convention Timetable Decisions problem (ECTTD).
\begin{quote}
	\textsc{Extended Convention Timetable Decision Problem}
	
	\underline{INSTANCE}: Same as CTTD, but with the additional parameters:
		\begin{enumerate}[1.]
			\setcounter{enumi}{4}
			\item a finite set $R$ of rooms;
			\item a collection $\{A_1,A_2,\cdots,A_r\}$, where $A_k \subseteq H$ (there are $r=|R|$ rooms and $A_k$ is the set of hours during which the $k$th room is available);
			\item a collection $\{S_1,S_2,\cdots,S_m\}$, where $S_l \subseteq R$ (there are $m$ talks and $S_m$ is the set of rooms that the $l$th presentation may be given in)
		\end{enumerate}
	\underline{QUESTION}: Does there exist a function 
		\begin{gather*}
			f(i,j,h,r) : \{1,\cdots,n\} \times \{1,\cdots,m\} \times H \times R \rightarrow \{0,1\}
		\end{gather*}
		(where $f(i,j,h,r)=1$ if and only if presenter $i$ gives talk $j$ during hour $h$ in room $r$) such that
		\begin{enumerate}[(a)]
			\item $f(i,j,h,r) = 1 \Rightarrow h \in P_i \cap T_j \cap A_r \land r \in S_j$ (the $i$th presenter, $j$th presentation and room $r$ are all available to be scheduled at hour $h$ and room $r$ is suitable for the $j$th presentation);
			\item $\sum\limits_{r \in R}\sum\limits_{h \in H} f(i,j,h,r) = G_{ij}$ for all $1 \le i \le n$ and $1 \le j \le m$ (the $i$th presenter was scheduled for the $j$th presentation the required number of times);
			\item $G_{ij} > 0 \land G_{i'j} > 0 \Rightarrow f(i,j,h,r)=f(i',j,h,r)$ for all $1 \le i,i' \le n$, $1 \le j \le m$, $h \in H$, and $r \in R$ (all presenters that are required to give a talk must be present at all instances of that talk);
			\item $\sum\limits_{r \in R}\sum\limits_{j=1}^m f(i,j,h,r) \le 1$ for all $1 \le i \le n$ and $h \in H$ (no presenter is giving more than one talk simultaneously);
			\item $\sum\limits_{j=1}^m f'(j,h,r) \le 1$ for each $h \in H$ and $r \in R$ where $f'(j,h,r) = 1 \iff \exists i \text{ with } 1 \le i \le n$ such that $f(i,j,h,r)=1$, and 0 otherwise (room $r$ is scheduled for at most one talk at hour $h$).
		\end{enumerate}
\end{quote}
\begin{cor}
ECTTD is NP-Complete.
\end{cor}
\begin{proof}
Let $L=(H,P,T,G)$ be an instance of CTTD. Let $R=T$, $A=\{H,H,\cdots,H\}$ with $|A|=R$ (i.e. each $A_k=H$), and  likewise let $S=\{H,H,\cdots,H\}$ with $|S|=|T|$. We now have an instance $L'=(H,P,T,G,R,A,S)$ of ECTTD (we note with slight amusement the embedded GRAPHS anagram). As with CTTD if an algorithm accepts this instance then there exists a function $f$ that meets the requirements (a-d).
\end{proof}

\subsection{Preference Convention Optimization Problem}
We have examined several different variations of scheduling problems as they relate to conventions; we now offer a final variation that will be the subject of study for the rest of the paper. In particular we are interested in not only finding an acceptable schedule but also a schedule that minimizes ``conflicts" to convention attendees. To this end, we will add a new piece of data to the puzzle, namely a set of convention attendees and their preferences for talks they would like to attend. Armed with this information we can attempt to find not only a schedule that meets the requirements of rooms, presenters, etc. but also one that is most acceptable to the attendees. We call this the Preference Convention Optimization problem (PCO).
\begin{quote}
	\textsc{Preference Convention Optimization Problem}
	
	\underline{INSTANCE}: Same as ECTTD, but with the additional parameters:
	\begin{enumerate}[1.]
		\setcounter{enumi}{7}
		\item a finite set $E = \{e_1, e_2, \cdots, e_t\}$ of attendees;
		\item a $t \times m$ 0-1 matrix $W$ ($W_{ej}$ indicates if the $e$th attendee would like to attend the $j$th talk).
	\end{enumerate}
	
	\underline{GOAL}: Find a function 
	\begin{gather*}
		f(i,j,h,r) : \{1,\cdots,n\} \times \{1,\cdots,m\} \times H \times R \rightarrow \{0,1\}
	\end{gather*}
	that minimizes the number of conflicts to the $W$ matrix subject to the constraints in ECTTD. We define a conflict as occurring when $f(i,j,h,r)=1, f(i',j',h,r')=1$ and there exists an $e$ such that $W_{ej}=1$ and $W_{ej'}=1$.
\end{quote}

\section{Integer programming models}

\subsection{Overview}
We will present integer programming models that can solve PCO and ECTTD. Integer programming is merely one way of solving these problems, other papers have described strictly combinational methods of solving similar problems \cite{cheng}. The data used to measure the models comes from a real convention held in 2013, which we shall refer to as PC2013. PC2013 had 195 presenters giving a total of 253 talks. Figure \ref{fig_pc2013_graph} helps illustrate the data we worked with: each vertex represents a talk and adjacent vertices share a common presenter and cannot be scheduled at the same time. By Theorem \ref{cttd_np} solving to CTTD is equivalent to asking if this graph admits an $h$-color (for some $h$, the number of timeslots available at the convention).
\begin{figure}[h!]
	\caption{Presenter conflicts that must be scheduled around in PC2013}
	\centering
		\includegraphics[scale=0.2]{penguiconconflict}
	\label{fig_pc2013_graph}
\end{figure}

\subsection{ECTTO}
The first problem we will solve with integer programming is the Extended Course Timetable Optimization (ECTTO) problem, which is the optimization version of ECTTD (i.e. find a schedule, not simply verify one exists). The goal of ECTTO is to find a scheduling function $f$. A possible problem is that $f$ can be very big.
\begin{gather*}
\text{size of } f = \text{\# of presenters } \times \text{ \# of talks } \times \text{ \# of hours } \times \text{ \# of rooms }
\end{gather*}
For PC2013, $\text{size of } f = 193 \times 253 \times 37 \times 15 = $ 27,421,440. Although not unheard of, this is a very large number of variables for the solver to work with. However, we know that several (nearly all) of these variables will be zero based on information we have at formulation time. For example, if a presenter $i$ doesn't give talk $j$, then $f(i,j,h,r) = 0$ for all $h \in H, r \in R$. We create a set of variables $\mathcal F \subseteq \{1,\cdots,n\} \times \{1,\cdots,m\} \times H \times R$ where $f_{i,j,h,r} \in \mathcal F \iff $ presenter $i$ gives talk $j$, the talk $j$, presenter $i$ and room $r$ are available at hour $h$, and room $r$ is suitable for the $j$th talk. In addition to this condition, we restrict the inclusion of variables in $\mathcal F$ to the intersection of the available hours of all \emph{co-presenters} (pairs of presenters that give the same talk), e.g. if co-presenters $i,i'$ have availability sets $\{h_2,h_3\}$ and $\{h_3,h_4\}$ (assuming room and talk availability is at least $\{h_2, h_3, h_4\}$) then only variables with $h=h_3$ for these co-presenters and talk will be included. In practice, the reduction of our solution space to only $\mathcal F$ gives us an enormous performance gain. For PC2013, this immediately reduced the number of variables down to 91,514 (a reduction of 99.997\%).

The variables in $\mathcal F$ can be thought of as a sparse representation of the domain of $f$. In our following formulation we will wish to ``iterate" over the dimensions of $f$, but only for those values with corresponding variables in $\mathcal F$. We will use a special convention to represent this. For example, when we say ``for all $j,h,r \in \mathcal F_i$, we mean all variables in $\mathcal F$ of the form $f_{i,j,h,r}$ with $i$ fixed.

In addition to $\mathcal F$ we will use the intermediate variable set $\mathcal G$ for convenience where $g_{j,h,r} \in \mathcal G \iff $ talk $j$ can be given by any presenter at hour $h$ in room $r$. We will now describe a formulation that implements each of the constraints on $f$ in ECTTO.
\begin{quote}
\textbf{ECTTO formulation}
\begin{flalign}
&\text{minimize: } 0& \\
&\text{subject to:}&\\
&\sum_{h,r \in \mathcal F_{i,j}} f_{i,j,h,r} = G_{ij}& \text{for every presenter $i$ and talk $j$}\label{ectto_ip_constraint_b}\\
&f_{i,j,h,r} - f_{i',j,h,r} = 0\label{ectto_ip_constraint_c_2}& \text{for every talk $j$ with co-presenters $i$, $i'$}\\
&\sum_{j,r \in \mathcal F_{i,h}} f_{i,j,h,r} \le 1& \text{for every presenter $i$ and hour $h$ }\label{ectto_ip_constraint_d}\\
&\left(\sum_{i \in \mathcal F_{j,h,r}} f_{i,j,h,r}\right) - U \times g_{j,h,r} \le 0& \text{for each $g_{j,h,r} \in \mathcal G$} \label{ectto_ip_constraint_e_1}\\
&\sum_{j \in \mathcal G_{h,r}} g_{j,h,r} \le 1& \text{for each hour $h$ and room $r$}\label{ectto_ip_constraint_e_2}\\
&\text{binary: } f_{i,j,h,r}, g_{j,h,r}
\end{flalign}
\end{quote}
The first requirement (a) of ECTTO merely enforces all availability and suitability sets. We implicitly enforce this in our model by considering only the variables in $\mathcal F$. As such, no specific constraints are needed in our model.

The second requirement (b) ensures that every presenter is scheduled for all of their talks, which we receive as parameter $G$ to ECTTO where $G_{ij}$ is the number of times presenter $i$ should give talk $j$. For each presenter $i$ and talk $j$, the sum of the times they are scheduled (over all hours and rooms) should be $G_{ij}$; this is constraint \ref{ectto_ip_constraint_b}.

To ensure requirement (c) we force that all co-presenters have the same schedule for their shared talk (constraint \ref{ectto_ip_constraint_c_2}).

Requirements (a)-(c) guarantee that all presenters are scheduled for their talks and that co-presenters are scheduled together. Requirement (d) makes sure that for presenters with multiple talks that these talks are at different hours (as we know a person cannot be in two places at once!) For each presenter $i$ and hour $h$, the sum of their schedule variables for their talks in all rooms must be less than one (constraint \ref{ectto_ip_constraint_d}).

The final requirement (e) makes sure that room scheduling is exclusive. If omitted then multiple talks could be legally scheduled in the same room at the same time. We would like to simply iterate over $\mathcal F$ for a particular pair of hour $h$ and room $r$, summing all of these together. If we didn't allow co-presenters (like TTD) then we could simply make this sum less than or equal to one. But, for talks with co-presenters this sum varies. To overcome this we create indicator variables $g_{j,h,r}$ where $g_{j,h,r}=1 \iff$ talk $j$ is scheduled at hour $h$ in room $r$. To do this we use a ``boolean-cast conversion trick" (constraint \ref{ectto_ip_constraint_e_1}). First, a large constant $U$ is picked, in this case we let $U=\text{size of } f$. For every pair triple $j,h,r$ we sum the values for all presenters. This sums over a subset of $f$, so it clearly must be less than $U$ since $f$'s values are 0-1. If the talk $j$ is not scheduled at hour $h$ in room $r$ then the left hand side of the constraint will be zero, and since we are minimizing $g_{j,h,r}$ will be zero. If the talk is scheduled then the right hand will be greater than zero, and so $g_{j,h,r}$ will be forced to one to ensure the entire expression is less than zero. Lastly, constraint \ref{ectto_ip_constraint_e_1} ensures that no room is multiply booked by checking the sum of $g_{j,h,r}$ for each pair $h,r$.

We solved this formulation with the open source IP solver CBC on PC2103 with varying numbers of talks pruned out to see how the model scales. The results are given in Figure \ref{ectto_run_time}.
\begin{figure}[h!]
	\caption{Run time of ECTTO model}
	\centering
		\input ecttd_model_run_time
	\label{ectto_run_time}
\end{figure}

\subsection{PCO}
We now turn our attention to the Preference Convention Optimization (PCO) problem. PCO adds an additional layer of complexity to ECTTO by including a matrix of preferences for attendees with the goal of minimizing the number of conflicts caused by concurrent talks. Through experimentation we have found that PCO greatly increases the difficulty of solving our convention scheduling problem. We first present our integer programming model for PCO.

The PCO model we present builds on our previous ECTTO model. In addition to the $G$ variables which collapse the four dimensional $f$ function down to three dimensions ($\text{talk} \times \text{hour} \times \text{room}$), all the while considering only those variables for which a feasible schedule is even possible given the different availability constraints, we will introduce four new sets for PCO. The first is $z$ which will collapse $g$ to two dimensions ($\text{talk} \times \text{hour}$). Next, we will expand $z$ to $z'$ which will have a three dimensional range from $\text{talk} \times \text{talk} \times \text{hour}$; it will represent if two talks $j, j'$ are both given concurrently at hour $h$. The final set $c$ has the same dimensions and will represent the conflicts (calculated from $W$, the preference matrix) if $j, j'$ are both scheduled at $h$. As with previous models, the corresponding script (e.g. $\mathcal Z$ for $z$) will represent the set of variables that are possible given availability constraints. The final set $w$ summarizes $W$ as a $\text{talk} \times \text{talk} \times \text{hour}$ matrix where each entry is the number of attendees who wish to attend both talks $j,j'$ for all $h$ when $j,j'$ can be given based on talk, presenter, and room availability. Formally, $w_{j,j',h} = |\{e_k \in E \; : \; e_k \text{ has } W_{kj} = 1 \text{ and } W_{kj'} = 1 \}|$ for each $h \in \mathcal{Z}_{j} \cap \mathcal{Z}_{j'}$. We will minimize our model on the sum all elements in $c$.
\begin{quote}
\textbf{PCO formulation}
\begin{flalign}
&\text{minimize: } \sum_{j,j',h \in \mathcal C} c_{j,j',h}& \label{pco_ip_objective} \\
&\text{subject to:}&\\
&\sum_{h,r \in \mathcal F_{i,j}} f_{i,j,h,r} = G_{ij}& \text{for every presenter $i$ and talk $j$}\label{pco_ip_constraint_b}\\
&f_{i,j,h,r} - f_{i',j,h,r} = 0\label{pco_ip_constraint_c_2}& \text{for every talk $j$ with co-presenters $i$, $i'$}\\
&\sum_{j,r \in \mathcal F_{i,h}} f_{i,j,h,r} \le 1& \text{for every presenter $i$ and hour $h$ }\label{pco_ip_constraint_d}\\
&\left(\sum_{i \in \mathcal F_{j,h,r}} f_{i,j,h,r}\right) - U \times g_{j,h,r} \le 0& \text{for each $g_{j,h,r} \in \mathcal G$} \label{pco_ip_constraint_e_1}\\
&\sum_{j \in \mathcal G_{h,r}} g_{j,h,r} \le 1& \text{for each hour $h$ and room $r$}\label{pco_ip_constraint_e_2}\\
&\left(\sum_{j,h \in \mathcal G_{r}} g_{j,h,r}\right) - U \times z_{j,h} \le 0& \text{for each room $r$}\label{pco_ip_constraint_z}\\
&\sum_{h \in \mathcal Z_{j}} z_{j,h} = G_{ij}& \text{for each talk $j$ \text{ and some presenter $i$}}\label{pco_ip_constraint_z_2}\\
&z_{j,h} + z_{j',h} - 2 \times z'_{j,j',h} \le 1& \text{for each hour $h$ and $j,j' \in \mathcal Z_h$ with $j \ne j'$}\label{pco_ip_constraint_z'} \\
&c_{j,j',h} - w_{j,j',h} \times z'_{j,j',h} = 0& \text{for all $j,j',h \in \mathcal {Z'}$} \label{pco_ip_constraint_w}\\
&\text{binary: } f_{i,j,h,r}, \; g_{j,h,r}, \; z_{j,h}, \; z'_{j,j',h}&\\
&\text{integer: } c_{j,j',h}&
\end{flalign}
\end{quote}
Constraints \ref{pco_ip_constraint_b} - \ref{pco_ip_constraint_e_2} are the as our model for ECTTO. Constraint \ref{pco_ip_constraint_z} begins to build the $z$ variables which will be 0-1 indicators of talk $j$ being given at hour $h$ via the same boolean cast mechanism described previously by collapsing the room entries for $j,h$ in $g$. To ensure that only the correct number of $z$s are set to one, constraint \ref{pco_ip_constraint_z_2} sums all hours $h$ for each talk $j$ and sets it equal to the number of times that talk $j$ was set to be given in the problem instance (the matrix $G$). It is of note that we pick \emph{any} presenter $i$'s entry in $G$ for talk $j$; although it is possible that a co-presenter $i'$ may have a different value for $G_{i'j}$ requirement (c) of PCO explicitly forbids this since it would be impossible for all co-presenters to be at all instances of a talk if they had different entries for their shared talk $j$, thus we can pick any presenter $i$.

The $z$ variables will now be used to generate $z'$, which will indicate if a pair of talks $j,j'$ are being given concurrently at hour $h$. Specifically, $z'_{j,j',h} = 1 \iff z_{j,h} = 1 \text{ and } z_{j',h} = 1$. Constraint \ref{pco_ip_constraint_z'} sets the difference of the sum of the $z$ variables for a particular triple $j,j',h$ and twice the corresponding $z'$ to being less than one. Alone this constraint does little; $z'_{j,j',h} = 1$ always satisfies this. Constraint \ref{pco_ip_constraint_w} sets $c_{j,j',h}$ to be equal to $z'_{j,j',h}$ times $w_{j,j',h}$ (the number of RSVP conflicts if $j$ and $j'$ are scheduled at $h$). The objective of the model is to minimize the sum of $c$ (Constraint \ref{pco_ip_objective}). Thus, $z'_{j,j'h}$ effectively controls if $c_{j,j',h}$ is non-zero and so the solver will attempt to minimize the instances where $z'_{j,j',h} = 1$; it only ``has" to be 1 when both $z_{j,h}$ and $z_{j',h}$ are 1, otherwise $z'{j,j',h}=0$ will also satisfy Constraint \ref{pco_ip_constraint_z'}.

To measure the efficiency of our model we tested it on our sample set PC2013. Attendance figures we recorded for each talk given at PC2013; a distribution of the attendance per talk is shown in Figure \ref{2013_attendance_distribution}.
\begin{figure}[h!]
\caption{Distribution of attendance at PC2013}
	\centering
		\input 2013_attendance_distribution
	\label{2013_attendance_distribution}
\end{figure}
The sum of all attendance counts was 4101 \cite{pc2013_attendance} for around 1000 unique attendees. For the purposes of testing our model created a $W$ such that $\sum\limits_{i=0}^{n} W_{ij}$ was equal to the attendance count for that talk $j$, i.e. we created an ``RSVP" for each real talk attendance at PC2013. Since individual attendee attendance wasn't tracked (only totals were) we took some liberties in distributing the RSVPs across the attendees in our model. We first randomly spread the RSVPs over the attendees using a uniform distribution; that is, if talk $j$ had an attendance of 24 in PC2013 then 24 attendees were randomly chosen to attend this talk. We solved our model with commercial solver Gurobi which returned a solution with an objective value of 0 after 64 seconds\footnote{PCO models were run on a machine with 4 12-core Intel Xeon E5-2695 CPUs running at 2.4 GHz with 96 GB of RAM. The authors would like to thank the SECS department at Oakland University for graciously providing the computation time for this paper. They would also like to thank Gurobi Optimization, Inc. for furnishing a free academic license of their Gurobi Optimizer.} i.e. a schedule with absolutely no attendee conflicts.

After finding a non-conflicting schedule for uniformly distributed random attendees that followed the attendance counts in PC2013 we turned our attention to how the solver would react when the random attendees were not distributed uniformly. Our intuition was that, like the actual attendance figures, the distribution of attendance per attendee would not be evenly spaced out; there would be some attendees who went to many talks, and some who went to only a few. We chose a normal distribution with $\mu = 500, \sigma = 100$. Even with an extremely powerful computer to run our model on and a state-of-the-art commercial solver we were unable to solve this instance after 24 hours. To help understand this phenomenon we created a half-sized problem (half as many talks and hours) and ran the model with decreasing values of $\sigma$ and found that the solving time exploded exponentially as $\sigma$ decreased; the average running time for $\sigma = 200$ was 19 seconds but increased to 17,012 seconds for $\sigma = 100$.

One of the reasons integer programming models can explode in running time is when the solver is unable to efficiently deal with symmetries in the model \cite{sherali}. An integer programming model is \emph{symmetric} if its variables can be permuted without changing the structure of the problem \cite{margot}. Our model exhibits high amounts of symmetry in relation to the scheduling of talks in rooms. If two rooms have the same availability and suitability set then permuting talk assignments among them in each hour produces no discernible change to the objective. It may be however that the solver will choose to branch early on in its branch-and-bound tree on these room assignments, leading to lots of unnecessary computation. In general it is difficult for the solver to detect that such variables ``really" represent the same thing, although there are several mechanisms for  determining and avoiding symmetry in solvers \cite{ostrowski}. However, it is easy for us to identify this symmetry and avoid it.

Two rooms will be said to be \emph{symmetric} if they have the same suitability and availability sets, i.e. for rooms $R_\alpha$ and $R_\beta$ we have that $R_\alpha$ and $R_\beta$ are symmetric if and only if
\begin{gather*}
A_\alpha = A_\beta \quad \text{and} \\
\{i \; | \; R_\alpha \in S_i \text{ for all } 1 \le i \le m\} = \{i \; | \; R_\beta \in S_i \text{ for all } 1 \le i \le m\}.
\end{gather*}
To break the symmetry we create room \emph{classes} which will represent several rooms with the same attributes. First, we make a new room set $R' = \{ \{r_1, r_2, \cdots, r_p \} \; | \; \text{all } r \in R \text{ are symmetric with each other} \}$. The corresponding new availability set $A'$ has simply the common availability set for each new $r' \in R'$. For the new suitability set $S'$ we replace each instance of $r$ with the room class that $r$ is a member of. When we solve our model talks will be booked to room classes, avoiding the symmetry that arises by having to consider two essentially ``equal" rooms separately. When our model is solved we will have bookings in room classes, and we can arbitrarily assign the talk to any room in that class. We must make only one adjustment to our model: Constraint \ref{pco_ip_constraint_e_2} in the PCO model ensures that each room has only one talk booked in it per hour. For our room classes we wish to relax this, requiring only that the number of bookings be at most the number of rooms in the class; this way, when we assign actual rooms from the solved model we can match talks to rooms in a one-to-one way. Formally, we will change Constraint \ref{pco_ip_constraint_e_2} to
\begin{gather*}
\sum_{j \in \mathcal G_{h,r}} g_{j,h,r} \le |r|.
\end{gather*}
It is easy to see that this model degenerates to our regular PCO model when no rooms are symmetric; in this case each room class would contain only one room. In practice the removal of the room symmetries led to around a 5x performance increase for our solver on PC2013 which contained only three room classes but had fourteen rooms (see Figure \ref{2013_normal_sigma_run_time}). 
\begin{figure}[h!]
\caption{Run time of PCO model with decreasing $\sigma$}
	\centering
		\input 2013_normal_sigma_run_time
	\label{2013_normal_sigma_run_time}
\end{figure}

\begin{thebibliography}{9}
\bibitem{even76}
  Even, S., A. Itai, and A. Shamir. 1976. On the complexity of timetable and multicommodity flow problems. SIAM Journal on Computing 5, (4) (12): 691-13
\bibitem{garey76}
  Garey, M., and D. Johnson. 1976. \underline{Computers and Intractability: A Guide to the Theory of NP-Completeness}. New York: W. H. Freeman
\bibitem{lovelace2010}
  Lovelace, A. 2010. On the complexity of scheduling university classes. M.S. in Computer Science Thesis. California Polytechnic State University: U.S.A.	
\bibitem{garey76_2}
  Garey, M., D. Johnson, and L. Stockmeyer. 1976. Some simplified NP-Complete graph problems. Theoretical Computer Science 1: 237-267
\bibitem{cheng}
  Cheng, E., Kleinberg, S., Kruk, S., Lindsey, W., and D. Steffy 2004. A strictly combinational approach to a university exam scheduling problem. Congressus Numerantium 167: 121-132
\bibitem{pc2013_attendance}
  Penguicon Programming Ops. \underline{http://penguicon.info/doku.php/programmingops?s[]=attendance}. Retried November 6, 2014
\bibitem{scip}
  Achterberg, T. 2009. SCIP: solving constraint integer programs. Mathematical Programming Computation, Volume 1, Number 1: 1-41
\bibitem{zimpl}
  Koch, T. 2004. Rapid Mathematical Programming. Ph.D. Thesis. Technische Universit{\"a}t Berlin. Germany
\bibitem{sherali}
  Sherali, H.D., and J.C. Smith. 2001. Improving Discrete Model Representations via Symmetry Considerations. Managements Science 47: 1396-1407.
\bibitem{margot}
  Margot, F. 2009. Symmetry in Integer Linear Programming. 2010. \underline{50 Years of Integer Programming 1958-2008}, Chapter 16: 647-681. Springer.
\bibitem{ostrowski}
  Ostrowski, J. 2008. Symmetry in Integer Programming. Ph.D. Thesis. Lehigh University: U.S.A. 
\end{thebibliography}

\end{document}